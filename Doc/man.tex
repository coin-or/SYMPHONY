\documentclass[twoside,11pt]{book}

\oddsidemargin 0in
\evensidemargin 0in
\topmargin 0in
\textwidth 6.5in
\newlength{\headwidth}
\setlength{\headwidth}{\textwidth}
\addtolength{\headwidth}{-4.4444pt}
\textheight 8.7in

\catcode`\@=11\relax   %allow @ in macro names

\if@twoside         % If two-sided printing.
\def\ps@headings{\let\@mkboth\markboth
\def\@oddfoot{}\def\@evenfoot{}%       No feet.
\def\@evenhead{
\begin{tabular}{@{}c@{}}
\hspace*{\headwidth} \\[-3ex]
\rm \thepage  \hfill  \sl \leftmark \\
\hline
\end{tabular}
} % Left heading.
\def\@oddhead{
%{\flushleft \sl \rightmark}
%\vspace*{-1ex}
%{\flushright \rm \thepage}
%\vspace*{-1ex}
%{\flushleft \underline{\hspace*{\headwidth}}}
\begin{tabular}{@{}c@{}}
\hspace*{\headwidth} \\[-3ex]
\sl \rightmark \hfill \rm \thepage \\
\hline
\end{tabular}
} % Right heading.
\def\sectionmark##1{\markboth {\uppercase{\ifnum \c@secnumdepth >\z@
    \thesection\hskip 1em\relax \fi ##1}}{}}%
\def\subsectionmark##1{\markright {\ifnum \c@secnumdepth >\@ne
          \thesubsection\hskip 1em\relax \fi ##1}}}
\else               % If one-sided printing.
\def\ps@headings{\let\@mkboth\markboth
\def\@oddfoot{}\def\@evenfoot{}%     No feet.
\def\@oddhead{
\begin{tabular}{@{}c@{}}
\hspace*{\headwidth} \\[-3ex]
\sl \rightmark \hfill \rm \thepage \\
\hline
\end{tabular}
} % Heading.
\def\sectionmark##1{\markright {\uppercase{\ifnum \c@secnumdepth >\z@
    \thesection\hskip 1em\relax \fi ##1}}}}
\fi

\catcode`\@=12\relax   %disable @ in macro names

\pagestyle{headings}

\markright{\bf \thesection}{}

\usepackage{epsfig, html}

\newcommand{\be}{\begin{enumerate}}
\newcommand{\ee}{\end{enumerate}}
\newcommand{\bc}{\begin{center}}
\newcommand{\ec}{\end{center}}
\newcommand{\bt}{\begin{tabular}}
\newcommand{\et}{\end{tabular}}
\newcommand{\bd}{\begin{description}}
\newcommand{\ed}{\end{description}}
\newcommand{\bi}{\begin{itemize}}
\newcommand{\ei}{\end{itemize}}
\newcommand{\bv}{\begin{verbatim}}
\newcommand{\ev}{\end{verbatim}}
\newcommand{\functiondef}[1]{\subsubsection{#1}}
\newcommand{\firstfuncdef}[1]{\subsubsection{#1}}
%begin{latexonly}
\renewcommand{\functiondef}[1]{\newpage \item[\Large $\triangleright$] {\bf
\Large #1}}
\renewcommand{\firstfuncdef}[1]{\item[\Large $\triangleright$] {\bf
\Large #1}}
%end{latexonly}
\newcommand{\describe}{\item[Description:] \hfill}
\newcommand{\args}{\item[Arguments:] \hfill}
\newcommand{\returns}{\item[Return values:] \hfill}
\newcommand{\postp}{\item[Post-processing:] \hfill}
\newcommand{\nopostp}{\item[Post-processing:] None \hfill}

\newcommand{\BB}{{\sc SYMPHONY}}
\newcommand{\TM}{{\sc TreeManager}}
\newcommand{\LP}{{\sc LP}}
\newcommand{\ra}{$\rightarrow$}
\newcommand{\bs} {\(\backslash\)}

%\htmladdtonavigation{\htmladdnormallink 
%       {\htmladdimg{contents_motif.gif}}{man.html}}

\begin{document}

%
%\ \\
%\vspace*{.5in}

\begin{rawhtml} <H1 ALIGN="CENTER"> 
\end{rawhtml}
\title{\huge {\bf \BB\ 4.0 User's Manual}
\thanks{This research was partially supported by NSF Grant
DMS-9527124 and Texas ATP Grant 97-3604-010}
%begin{latexonly}
\vskip 1.7in
\LARGE 
SYMPHONY Developed By \\ 
\vskip .2in
T.K. Ralphs\thanks{Department of Industrial and
Systems Engineering, Lehigh University, Bethlehem, PA 18017, {\tt
tkralphs@lehigh.edu}, {\tt http://www.lehigh.edu/\~{ }tkr2}} \\
L. Lad\'anyi\thanks{Department of Mathematical Sciences, 
IBM T.J. Watson Research Center, Yorktown Heights,
NY 10598} \\
\vskip .5in
Interactive Graph Drawing \\
Software By \\
\vskip .2in
M. Es\"o\thanks{Department of Mathematical Sciences, 
IBM T.J. Watson Research Center, Yorktown Heights,
NY 10598}
\vskip .5in
%end{latexonly}
}
\maketitle
\begin{rawhtml} </H1> 
\end{rawhtml}
\begin{htmlonly}
\begin{center}
{\LARGE 
\vskip 2.2in
\begin{rawhtml} <P ALIGN="CENTER"> <STRONG> <FONT SIZE="4">
\end{rawhtml}
SYMPHONY Developed By \\ 
\vskip .2in
T.K. Ralphs \\
L. Lad\'anyi \\
\begin{rawhtml} </P> </FONT> </STRONG> 
\end{rawhtml}
\vskip .5in
\begin{rawhtml} <P ALIGN="CENTER"> <STRONG> <FONT SIZE="4">
\end{rawhtml}
Interactive Graph Drawing \\
Software By \\
\vskip .2in
M. Es\"o}
\begin{rawhtml} </P> </FONT> </STRONG> 
\end{rawhtml}
\begin{rawhtml} <P ALIGN="CENTER"> <STRONG> <FONT SIZE="3">
PDF version available <A
HREF="ftp://branchandcut.org/pub/reference/SYMPHONY-4.0-Manual.pdf">here</A>
</FONT></STRONG></P>
\end{rawhtml}
\begin{rawhtml} </P> </FONT> </STRONG> 
\end{rawhtml}
\begin{rawhtml} <P ALIGN="CENTER"> <STRONG> <FONT SIZE="3">
A revised version of 
Sections 3-8 of this manual
now appears in the Springer-Verlag book <br>
<i><a href="http://link.springer.de/link/service/series/0558/tocs/t2241.htm">
Computational Combinatorial Optimization </i></a>
edited by M. Juenger and D. Naddef.
</FONT></STRONG></P>
\end{rawhtml}

\end{center}
\end{htmlonly}

\newpage

\thispagestyle{empty}

\ \\
%begin{latexonly}
\vspace*{3in}
%end{latexonly}
\begin{rawhtml} <P ALIGN="CENTER"> <STRONG> 
\end{rawhtml}
\begin{center}
{\copyright 2000-2003 Ted Ralphs\footnote{A revised version of 
Sections 3-8 of this manual
now appears in the Springer-Verlag book {\em Computational Combinatorial 
Optimization} 
edited by M. J\"unger and D. Naddef, see 
http://link.springer.de/link/service/series/0558/tocs/t2241.htm}.}
\end{center}
\begin{rawhtml} </P> </STRONG> 
\end{rawhtml}

\newpage

\tableofcontents

\newpage

\setcounter{page}{1}

\section{Introducing SYMPHONY 5.0}
\label{whats-new}

Welcome to the SYMPHONY user's manual. Whether you are a new user or simply
upgrading to version 5.0, this manual will help you get started with what we
hope you will find to be a very useful framework for solving mixed-integr
linear programs either using the generic tools provided or by developing a
custom branch, cut, and price algorithm. There have been some very significant
developments since the last version of SYMPHONY was released. IN particular,
SYMPHONY is now a callable library with an interface whose look and feel is
similar to other popular solvers. This change allows SYMPHONY to be used in a
variety of new and powerful ways that were not possible before. For existing
users, there have been a few minor changes to the API needed to make SYMPHONY
thread-safe. Code written for previous versions of SYMPHONY will have to be
ported. Instructions for porting from previous version are contained in the
file \texttt{SYMPHONY-5.0/README-5.0}. As always, these changes have
undoubtedly introduced bugs. There are now an even larger number of
configurations in which SYMPHONY can be used and we have tested many of them,
but it is simply not possible to test them all. Please keep this in mind and
report all bugs that you find. Among the new enhancements and features are:

\begin{itemize}

\item SYMPHONY is now a C callable library with an interface whose look and
feel is similar to other popular solvers. This interface works for SYMPHONY's
built-in generic MILP solver, as well as any customized algorithm developed by
implementing one or more of SYMPHONY's user callback functions. The interface
is exactly the same for both sequential and parallel versions of the code.

\item The callable library also has a C++ interface conforming to COIN-OR's
Open Solver Interface standard for accessing LP and MILP solvers.

\item SYMPHONY has been made thread-safe in order to allow multiple
environments to be opened within a single executable.

\item It is now possible to stop SYMPHONY during the solution process and then
restart the computation later, even after modifying the problem data. The user
can also save warm start inforation outside the solver environment and then
reload it later into a different environment, in much the same way as can be
done with a simplex-based linear programming solver. This allows the user to
efficiently implement procedures, such as those for multi-criteria
optimization, in which a series of similar MILPs must be solved.

\item Along with the ability to perform warm starts, the user call also define
permanent cut pools that persist between solver calls. This is useful for
situations in which a series of MILPs needs to be solved and the cuts
generated during one solution call are still valid during later calls.

\item SYMPHONY now has the ability to enumerate the efficient solutions of a
bicriteria MILP if the user specifies a second objective function. This is
done using a new algorithm described in \cite{??} and takes advantage of the
warm starting capabilities of SYMPHONY.

\item SYMPHONY has a very rudimentary to perform sensitivity analysis for
MILP. This capability is till very much in the development stages, but is
present in version 5.0.

\end{itemize}

\section{How to Use This Manual}

The manual is divided into seven chapters. The first is the introduction,
which you are reading now. Chapter \ref{SYMPHONY-background} contains
background information. Those not familiar with the basic methodology of
branch, cut, and price should read these sections, especially Section
\ref{B&C-intro}, where we briefly describe the techniques involved. Chapter
\ref{API-overview} contains an overview of the API, both for the callable
library and for the user callback functions. Chapter \ref{SYMPHONY-design}
contains further depth and a more complete description of the design and
implementation of SYMPHONY. In Section \ref{design}, we describe the overall
design of without reference to the implementational details and with only
passing reference to parallelism. In Section \ref{modules}, we discuss the
details of the implementation. In Section \ref{parallelizing}, we briefly
discuss issues involved in parallel execution of SYMPHONY. It is not necessary
to read Chapters \ref{SYMPHONY-background} and \ref{SYMPHONY-design} before
undertaking development of a SYMPHONY application. Chapter
\ref{getting-started} describes how to install and compile SYMPHONY. Many
users will want to go straight to this section of the manusal to get started
quickly. Chapter \ref{SYMPHONY-development} describes in detail how to develop
a custom application using SYMPHONY. For those who are familiar with branch
and cut and want to get started quickly, proceed directly to to Section
\ref{getting_started} for information on getting started. Chapter
\ref{SYMPHONY-reference} contains reference material. Section
\ref{callable-library} contains a description of the interface for the
callable library, both for C and C++ environments. Section \ref{API} contains
a description of the user callback functions. SYMPHONY's parameters are
described in Section \ref{params}. Please note that for reference use, the
HTML version of this manual may be more practical, as the embedded hyperlinks
make it easier to navigate.


%\newpage

%\thispagestyle{empty}

%\ \\
%%begin{latexonly}
%\vspace*{3in}
%\begin{center}
%{\large PART II: Development}
%\end{center}
%%end{latexonly}

%\newpage

\chapter{Development}

\section{Quick Start Guide}
\label{getting_started}
SYMPHONY Version 4.0 is a powerful environment for implementing custom branch,
cut, and price algorithms. The subroutines in the \BB\ library comprise a
state-of-the-art MILP solver designed to be modular and easy to customize for
various problem settings. All internal library subroutines are generic---their
implementation does not depend on the the problem-setting. As of Version 4.0,
SYMPHONY works out of the box as a generic MILP solver, with the capability to
read both MPS files and GMPL (a subset of AMPL) files and solve the described
mixed integer programs. To customize SYMPHONY, various user subroutines can be
written and parameters set that modify the default behavior of the
algorithm. The API for these subroutines is described in this manual and files
containing function stubs are provided. As an example, by replacing the
default I/O subroutine, one can easily modify the solver so that it reads in
problem instances in a custom format (such as the TSPLIB format for specifying
traveling salesman problem instances).

The vast majority of the computation takes place within a ``black box,'' of
which the user need have no knowledge. \BB\ performs all the normal functions
of branch, cut, and price---tree management, LP solution, cut pool management,
as well as inter-process or inter-thread communication. Solvers can be built
in a wide variety of configurations, ranging from fully parallel to completely
sequential, depending on the user's needs. The library runs serially on almost
any platform, and can also run in parallel in either a fully distributed
environment (network of workstations) or shared-memory environment simply by
changing a few options in the makefile. To run in a distributed environment,
the user must have installed the {\em
\htmladdnormallink{Parallel Virtual Machine}{http://www.ccs.ornl.gov/pvm/}}
(PVM), available for free from Oak Ridge National Laboratories.
To run in a shared memory environment, the user must have installed an
OpenMP compliant compiler. A cross-platform compiler called {\em
\htmladdnormallink{Omni}{http://phase.etl.go.jp/Omni/}}, which uses 
{\tt cc} or {\tt gcc} as a back end, is available for free download
\begin{latexonly}
at {\tt http://phase.etl.go.jp/Omni/}
\end{latexonly}. For other options, visit
\texttt{\htmladdnormallink{http://www.openmp.org}{http://www.openmp.org}}.

New in Version 4.0 is the extensive use of the
\htmladdnormallink{COIN}{http://www.coin-or.org} optimization libraries.
SYMPHONY now uses COIN's OSI to interface with most commonly available LP
solvers. Generic cuts are generated using COIN's Cut Generator Library
(CGL) and the COIN LP Solver (CLP) can be used through the OSI to solve the LP
relaxations. The COIN MPS reader is used to read in MILP instances in MPS
format. For a summary of what else is new, see Section \ref{whats-new}. The
improvements have required a number of changes to the API. Code written for
previous versions of SYMPHONY will be broken, but not too badly. Instructions
for porting from previous version are contained in the file
\texttt{SYMPHONY-4.0/README-4.0}.

This section of the manual is concerned with the detailed
specifications needed to develop an application using \BB. It is
assumed that the user has already read the first part of the manual, which
provides a high-level introduction to parallel branch, cut, and price
and the overall design and use of \BB. 

%%%%%%%%%%%%%%%%%%%%%%%%%%%%%%%%%%%%%%%%%%%%%%%%%%%%%%%%%%%%%%%%%%%%%%%%%%%%%

\subsection{Compiling the Sample Application with Unix Operating Systems}
\label{getting_started_unix}

Here is a sketch outline of how to get started with SYMPHONY in Unix. This is
basically the same information contained in the README file that comes with
the distribution and will lead you through the steps required to compile
SYMPHONY as a generic MILP solver that can then be customized by filling out
the functions provided in the user interface files. For more information, see
Section \ref{advanced-compilation}. Because SYMPHONY is intended to run over
nonhomogeneous networks of workstations, installation is not fully automated,
but requires the user to make minor edits to the makefile.  With this setup,
compilation for multiple architectures and configurations can be performed in
a single directory without reconfiguring or ``cleaning.'' This is convenient
on nonhomogeneous networks, but it means that you might need to edit the
makefiles to get SYMPHONY to compile. For the casual user, this editing is
limited to providing some path names.

\subsubsection{Preparing for Sample Compilation.}

\begin{itemize}

        \item Download the file {\tt SYMPHONY-4.0.tgz}.

        \item Unpack the distribution with {\tt tar -xzf
        SYMPHONY-4.0.tgz}. This will create a subdirectory called
        {\tt SYMPHONY-4.0/} containing the distribution.
        
        \item Edit the makefile ({\tt SYMPHONY-4.0/Makefile}) to reflect your
              environment. This involves specifying the LP solver to be used
              and the paths to various libraries and include files. Only minor
              edits should be required. An explanation of what has to be set
              is contained in the comments in the makefile. 

        \item To use many of the new capabilities of SYMPHONY, you must have
	      installed the
	      \htmladdnormallink{COIN}{http://www.coin-or.org/}
	      optimization libraries \begin{latexonly} COIN optimization
	      libraries, available from \texttt{http://www.coin-or.org}
	      \end{latexonly}.  Specifically, you should install OSI, CGL, and
	      the Coin utilities (in the \texttt{COIN/Coin directory}). If you
	      want to use the COIN LP solver, you should also download CLP.
	      The path to the COIN libraries must be specified in
	      \texttt{SYMPHONY-4.0/Makefile}.

        \item If you wish to read GMPL/AMPL files, you will have to install
	      the Gnu Linear Programming Kit (GLPK), which contains a parser
	      for GMPL/AMPL files. The path to the GLPK libraries must be
	      specified in \texttt{SYMPHONY-4.0/Makefile}.

\end{itemize}
        
\subsubsection{Compiling the Sequential Version.}

\begin{itemize}
        \item Change to the \texttt{SYMPHONY-4.0/USER} directory and type
\texttt{ make}. This will first make the
\BB\ library (sequential version). After making the \BB\ library, make will
compile the user customization library and link the executable. The name of
the executable is specified in the file \texttt{SYMPHONY-4.0/USER/Makefile}
and is \texttt{mip} by default. The executable is installed in the directory
\texttt{SYMPHONY-4.0/USER/bin.\$(ARCH)/\$(LP\_SOLVER)}, where \texttt{ARCH} is 
the current architecture and \texttt{LP\_SOLVER} is the current LP solver, as 
specified in the makefile. The makefile can be modified to enable parallel 
execution of the code (see below).

\item After the SYMPHONY library is compiled, you are free to type {\tt make
        clean} if you want to save disk space. You should only have to
        remake the library if you change something in \BB's internal files. To
        clean the user files, type {\tt make clean\_user}.

\item To test SYMPHONY, a sample MPS file called \texttt{sample.mps} is included 
with the distribution. To specify the file name, use the \texttt{-F} command-line
option, i.e., type \texttt{bin.\$(ARCH)/\$(LP\_SOLVER)/mip -F sample.mps} in the
\texttt{SYMPHONY-4.0/USER} directory. To obtain more MPS data files for further
testing, download the MIPLIB library.

\item That's it! Now you are ready to develop your own application by modifying
the files in the \texttt{SYMPHONY-4.0/USER} directory and it's
subdirectories. See Section \ref{development} for more information.

\end{itemize}

\subsubsection{Compiling the Shared Memory Version.}

Please note that the shared-memory parallel version has not been tested in
Version 4.0 and may be broken. Please let me know if you want to use it and I
will get it working. 

\begin{itemize}
        \item To compile a shared memory version, obtain an OpenMP
        compliant compiler, such as \htmladdnormallink{Omni}
        {http://phase.etl.go.jp/Omni/}
\begin{latexonly} 
        (free from {\tt http://phase.etl.go.jp/Omni})
\end{latexonly}. 
        Other options are listed at \htmladdnormallink{the OpenMP Web
        site}{http://www.openmp.org}
\begin{latexonly}
        ({\tt http://www.openmp.org})
\end{latexonly}.

        \item Follow the instructions above for configuring the makefile. Set
the variable {\tt CC} to the compiler name in the makefile and compile as
above. Note that if you have previously compiled the sequential version, then
you should first type {\tt make clean\_all}, as this version uses the same
directories.  With one thread allowed, it should run exactly the same as the
sequential version so there is no need to compile both versions.

        \item Voila, you have a shared memory parallel solver. As above, to
test SYMPHONY, a sample MPS file called \texttt{sample.mps} is included with the
distribution. To specify the file name, use the \texttt{-F} command-line option,
i.e., type \texttt{bin.\$(ARCH)/\$(LP\_SOLVER)/mip -F sample.mps} in the
\texttt{SYMPHONY-4.0/USER} directory. To obtain more MPS data files for further
testing, download the MIPLIB library.

\item That's it! Now you are ready to develop your own application by modifying
the files in the \texttt{SYMPHONY-4.0/USER} directory and it's
subdirectories. See Section \ref{development} for more information.

\end{itemize}

\subsubsection{Compiling the Distributed Version.}

Please note that the distributed-memory parallel version has not been tested
in Version 4.0 and may be broken. Please let me know if you want to use it and
I will get it working.

\begin{itemize}
        \item If you wish to compile a distributed version of the code, obtain
and install the {\em \htmladdnormallink{Parallel Virtual
Machine}{http://www.csm.ornl.gov/pvm/}} (PVM) software, available for free
from Oak Ridge National Laboratories
\begin{latexonly}
        at {\tt http://www.ccs.ornl.gov/pvm/}
\end{latexonly}. 
        See Section \ref{PVM} for more notes on using PVM.
        
        \item In \texttt{SYMPHONY-4.0/Makefile}, be sure to set the {\tt
        COMM\_PROTOCOL} to {\tt PVM}. Also, in
        \texttt{SYMPHONY-4.0/USER/Makefile}, change one or more of {\tt
        COMPILE\_IN\_TM}, {\tt COMPILE\_IN\_LP}, {\tt COMPILE\_IN\_CG}, and
        {\tt COMPILE\_IN\_CP}, to {\tt FALSE}, or you will end up with the
        sequential version. Various combinations of these variables will give
        you different configurations and different executables. See Section
        \ref{configuration} for more info on setting them. Also, be sure to
        set the path variables in the makefile appropriately so that 
	\texttt{make} can find the PVM library.

        \item Type {\tt make} in the \texttt{SYMPHPONY-4.0/USER} directory to
        make the distributed libraries. As in Step 1 of the sequential
        version, you may type {\tt make clean} after making the
        library. It should not have to remade again unless you modify
        \BB's internal files.

        \item After the \BB\ libraries, user code will be compiled and
        required executables linked.

        \item Make sure there are links from your
        \texttt{\$PVM\_ROOT/bin/\$PVM\_ARCH/} directory to each of the
        executables in the
        \texttt{SYMPHONY-4.0/USER/bin.\$(ARCH)/\$(LP\_SOLVER)} directory. 
	This is required by PVM.

        \item Start the PVM daemon by typing {\tt pvm} on the command line
        and then typing {\tt quit}.

        \item As above, test SYMPHONY using the sample MPS file called
        \texttt{sample.mps} is included with the distribution. To specify the
        file name, use the \texttt{-F} command-line option, i.e., type
        \texttt{bin.\$(ARCH)/\$(LP\_SOLVER)/mip -F sample.mps} in the
        \texttt{SYMPHONY-4.0/USER} directory. To obtain more MPS data files
        for further testing, download the MIPLIB library.

        \item That's it! Now you are ready to develop your own application by
        modifying the files in the \texttt{SYMPHONY-4.0/USER} directory and
        it's subdirectories. See Section \ref{development} for more
        information.

\end{itemize}

%\noindent This should result in the successful compilation of the sample
%application. Once you have accomplished this much, you are well on
%your way to having an application of your own. Don't be daunted by the
%seemingly endless list of user function that you are about to
%encounter. Most of them are optional or have default options. If you
%get lost, consult the source code for the sample application to see
%how it's done.

\subsection{Compiling the Sample Application with Microsoft Windows}
\label{getting_started_windows}

Here is a sketch outline of how to compile SYMPHONY in MS Windows. Direct
support is provided for compilation with MS Visual Studio 6.0. Compilation for
other compilers should also be possible. Note that the Windows version has
some limitations. Detailed timing information is not currently provided.
Support is only provided for running in sequential mode at this time.

First, download \texttt{SYMPHONY-4.0.zip} and unzip the archive. This will
create a subdirectory called \texttt{SYMPHONY-4.0$\backslash$} containing all
the source files. You now have two options. You can either compile on the
command-line, using the MSVC++ makefile called
\texttt{SYMPHONY-4.0$\backslash$WIN32$\backslash$user.mak} or you
can use the provided projects and workspaces. Compiling on the command-line is
somewhat easier since it requires only editing the makefile and typing a
single command.

\subsubsection{Using the NMAKE Utility}

\begin{itemize}

\item Edit the file
\texttt{SYMPHONY-4.0$\backslash$USER$\backslash$WIN32$\backslash$user.mak} to
reflect your environment. This involves specifying the LP solver to be used
and various paths. Only minor edits should be required. An explanation of what
has to be set is contained in the comments in the makefile.

\item To use many of the new capabilities of SYMPHONY, you must have installed
the \emph{\htmladdnormallink{COIN}{http://www.ccs.ornl.gov/pvm/}} optimization
libraries 
\begin{latexonly} 
COIN optimization libraries, available from
\texttt{http://www.coin-or.org} 
\end{latexonly}. Specifically, you should
install OSI, CGL, and the Coin utilities (in the COIN/Coin directory). If you
want to use the COIN LP solver, you should also download CLP. The path to the
COIN libraries must be specified in
\texttt{SYMPHONY-4.0$\backslash$USER$\backslash$WIN32$\backslash$user.mak}.

\item If you wish to read GMPL/AMPL files, you will have to install the Gnu
Linear Programming Kit (GLPK), which contains a parser for GMPL/AMPL
files. The path to the GLPK libraries must be specified in
\texttt{SYMPHONY-4.0$\backslash$USER$\backslash$WIN32$\backslash$user.mak}.

\item Once configuration is done, type \texttt{nmake /f user.mak} at the
command prompt in the \\
\texttt{SYMPHONY-4.0$\backslash$USER$\backslash$WIN32} directory. The
executable \texttt{symphony.exe} will be created in the
\texttt{SYMPHONY-4.0$\backslash$USER$\backslash$WIN32$\backslash$Debug}  
directory.

\item To test the executable, type \texttt{symphony.exe -F ..$\backslash$..$\backslash$test.mps} at a command 
prompt in the
\texttt{SYMPHONY-4.0$\backslash$USER$\backslash$WIN32$\backslash$Debug} 
directory.

\end{itemize}

\subsubsection{Using the MSVC++ Workspace}

\begin{itemize}

\item In MS Visual C++ 6.0, open the workspace
\texttt{SYMPHONY-4.0$\backslash$USER$\backslash$WIN32$\backslash$user.dsw}. 
Note that there are two projects, one called \texttt{symphony} and one called
\texttt{user}.  The \texttt{symphony} project compiles the source of the internal
library. The \texttt{user} project compiles a separate library containing the
user-defined functions that can be used to customize the solver.

\item To use many of the new capabilities of SYMPHONY, you must have installed
the \htmladdnormallink{COIN}{http://www.ccs.ornl.gov/pvm/} optimization
libraries 
\begin{latexonly} 
COIN optimization libraries, available from
\texttt{http://www.coin-or.org} 
\end{latexonly}. Specifically, you should
install OSI, CGL, and the Coin utilities (in the \texttt{COIN/Coin}
directory). If you want to use the COIN LP solver, you should also download
CLP. The default location for COIN is C:$\backslash$COIN$\backslash$.

\item By default, SYMPHONY is set up to use the OSI CPLEX interface. To see this
check the following settings:

\begin{itemize}

\item \texttt{\_\_OSI\_CPLEX\_\_} should be defined in the preprocessor
definitions of both \texttt{symphony} and \texttt{user} projects (to see that,
right-click on one of the projects, and then choose \texttt{Settings --> C/C++
--> Preprocessor} in the category drop-down menu).

\item Paths to the include files of COIN utilities (Coin), OSI, OSI\_CPLEX, and
CPLEX are specified in the same settings window as for the preprocessor
definitions. Note that the Coin, OSI, and OSI\_CPLEX include directories are
assumed to be in \texttt{C:$\backslash$COIN$\backslash$Coin},
\texttt{C:$\backslash$COIN$\backslash$Osi},
\texttt{C:$\backslash$COIN$\backslash$Osi$\backslash$OsiCpx} and 
\texttt{C:$\backslash$ILOG$\backslash$cplex81$\backslash$include$\backslash$ilcplex}
directories, respectively. If they are not, make sure that you have set the
correct paths in both projects before compiling.

\item The \texttt{symphony} project is dependent on the \texttt{user} project
(see the dependencies in \texttt{Project --> Dependencies}) and it includes the
necessary libraries: \texttt{user}, \texttt{osiCpxLib}, \texttt{osiLib},
\texttt{coinLib} and \texttt{cplex81} (solver library).

\end{itemize}

If you want to use the native CPLEX interface (without downloading COIN) or a
solver other than CPLEX:

\begin{itemize}

\item If you want to use another OSI interface, change the preprocessor
definition from \texttt{\_\_OSI\_CPLEX\_\_} to \texttt{\_\_OSI\_XXX\_\_} ,
where \texttt{XXX} is replaced by the desired solver's acronym (e.g.,
\texttt{\_\_OSI\_CLP\_\_}, \texttt{\_\_OSI\_GLPK\_\_},
\texttt{\_\_OSI\_OSL\_\_}, etc.). Otherwise, change it to either
\texttt{\_\_CPLEX\_\_} or \texttt{\_\_OSL\_\_} in both projects.

\item Change the path definitions of the include files: for instance, if you want
to use \texttt{\_\_OSI\_OSL\_\_}, define
\texttt{C:$\backslash$COIN$\backslash$Osi$\backslash$OsiOsl} and \texttt{C:$\backslash$Program
Files$\backslash$IbmOslV3Lib$\backslash$osllib} (assuming it is installed
there) instead of the CPLEX path definitions. If you want to use the native
OSL interface, delete all the path definitions (you are not required to have
COIN or OSI), and just add the path definition for the OSL include files.

\item Add the appropriate libraries to the \texttt{symphony} project. For
instance, if you want to use \texttt{\_\_OSI\_OSL\_\_}, then add the
\texttt{osiOslLib} and \texttt{oslmd6030} (OSL solver lib file) library files
after deleting \texttt{osiCpxLib} and \texttt{cplex81} libraries from the
\texttt{symphony} project (you can do that by right-clicking on the
\texttt{symphony} project and then choosing --> Add files to project with
\texttt{library files} selected as the file type). If 
you want to use the native OSL interface, then delete all the libraries
(except the \texttt{user} library ) from the \texttt{symphony} project and
just add the \texttt{oslmd6030} library file for it is the unique solver
library file we need now.

\end{itemize}

\item By default, SYMPHONY is also set up to use the COIN CGL library for
generating cuts. To use CGL, the \texttt{symphony} project has the
\texttt{ADD\_CGL\_CUTS} preprocessor definition, the path to
\texttt{C:$\backslash$COIN$\backslash$Cgl$\backslash$} (be sure that this path
directs SYMPHONY to the include directory of CGL) and a dependence on the
\texttt{cgllib} library. If you also don't want to use the CGL library, simply
delete the \texttt{ADD\_CGL\_CUTS} preprocessor definition, the CGL path
definitions and the \texttt{cgllib} library from the \texttt{symphony}
project.

\item DO NOT CHANGE COMPILER DEFINES NOT RELATED TO THE LP SOLVER. Important note
for OSL users: when using OSL in Windows, you must also add \texttt{OSLMSDLL}
to the list of definitions.

\item Note that there are a number of additional preprocessor definitions that
control the functionality of SYMPHONY. These definitions are described in
\texttt{SYMPHONY-4.0/Makefile}, a Unix-style makefile included with the
distribution. To enable the functionality associated with a particular
definition, simply add it to the list of definitions, as above.

\item You must also be sure to have any \texttt{.dll} files required for your
LP solver to be in your search path. Either move the required \texttt{.dll} to
the directory containing symphony.exe or add the path to the \texttt{PATH}
Windows environment variable.

\item Once you have the proper settings for your LP solver, choose \texttt{Build
symphony.exe} from the \texttt{Build menu}. This should successfully build
the executable.

\item To test the executable, right click on the \texttt{symphony} project, go
to the \texttt{Debug} tab and set the program arguments to \texttt{-F
test.mps} Note that command-line switches are Unix-style.

\item Now choose \texttt{Execute} from the build menu and the solver
should solve the sample problem.

\end{itemize}

Note that there is some functionality missing from the Windows version. Most
prominently, the timing functions do not work. This functionality
should be easy to add---let me know if you are interested in doing it and I
will give you all the help I can. In addition, the Windows version will only
run in sequential mode for a variety of reasons. However, it should be
relatively easy to get it running in parallel if you can get PVM working under
Windows. Let me know if you are interested.

\subsection{Customizing SYMPHONY}

\subsubsection{Unix}

Developing a simple branch and bound solver that can read in instance data in
a custom format is straightforward using the template provided in the
\texttt{SYMPHONY-4.0/USER/} directory. The simplest approach is to develop a 
sequential application, sharing a single user data structure among all the
modules. Using this approach, an application can be developed by filling in
only a few simple functions. Here are four steps to make the most common
custom modifications.

\begin{enumerate}

\item Decide what run-time parameters you might need and what data has 
to be stored to describe your instance. Edit the file
\texttt{SYMPHONY-4.0/USER/include/user.h}, adding the appropriate fields to the
\texttt{user\_parameter} and \texttt{user\_problem} structure to store the
relevant values.

\item Now edit the file \texttt{SYMPHONY-4.0/USER/Master/user\_master.c}. Only two
function needs to be filled out initially in order to read in your data files
in a custom format. The function \texttt{user\_io()} is where you can read in
the instance data, if it is in something other than MPS format (a TSPLIB file,
for example). Note that the name of the file to read the instance data from
can be specified on the command line using the \texttt{-F} option by
default. Also, you'll probably need to modify the
\texttt{user\_initialize\_root\_node()} function to set up the root node.

\item Edit the file \texttt{USER/LP/user\_lp.c}. The function
\hyperref{{\tt user\_create\_subproblem()}} {{\tt user\_create\_subproblem()} (see
Section }{)}{user_create_subproblem} must be filled out. This is the most
important function, as it sets up the initial LP relaxation in each search
node.

\item You may also want to add some problem-specific cutting planes by modifying
the \texttt{user\_find\_cuts()} function.

\end{enumerate}

Now compile by by typing \texttt{make} (assuming that the make
file has already been modified appropriately). That's it! You should now have
a working branch and bound solver. Next you can work on filling in more
advanced functions and adding cutting planes.

\subsubsection{Microsoft Windows}
Developing a simple branch and bound solver that can read in instance data in
a custom format is straightforward using the template provided in the
\texttt{SYMPHONY-4.0$\backslash$USER$\backslash$} directory. The simplest approach is to develop a
sequential application, sharing a single user data structure among all the
modules. Using this approach, an application can be developed by filling in
only a few simple functions. Here are some steps to make the most common
custom modifications.

\begin{enumerate}

\item Decide what run-time parameters you might need and what data has 
to be stored to describe your instance. Edit the file
\texttt{USER$\backslash$include$\backslash$user.h}, adding the appropriate
fields to the \texttt{user\_parameter} and \texttt{user\_problem} structure to
store the relevant values.

\item Now edit the file
\texttt{USER$\backslash$Master$\backslash$user\_master.c}. Only two
function needs to be filled out initially in order to read in your files in a
custom data format. The function \texttt{user\_io()} is where you can read in
the instance data, if it is in something other than MPS format (a TSPLIB file,
for example). Note that the name of the file to read the instance data from can
be specified on the command line using the \texttt{-F} option by
default. Also, you'll probably need to modify the
\texttt{user\_initialize\_root\_node()} function to set up the root node.

\item Edit the file \texttt{USER$\backslash$LP$\backslash$user\_lp.c}. The
function \hyperref{{\tt user\_create\_subproblem()}} {{\tt
user\_create\_subproblem()} (see Section }{)}{user_create_subproblem} must be
filled out. This is the most important function, as it sets up the initial LP
relaxation in each search node.

\item You may also want to add some problem-specific cutting planes by modifying
the \texttt{user\_find\_cuts()} function.

\end{enumerate}

Now build again and you have a working branch and bound solver! Next you can
work on filling in more advanced functions and adding cutting planes. See the
documentation for more information on this.

\subsection{Sample Applications}

There are now a number of sample applications available as examples of how
to do development with SYMPHONY. These include solvers for the matching
problem, the set partitioning problem (simple and advanced versions), the 
vehicle routing and traveling salesman problems, and the mixed postman
problem. These applications are distributed as separate packages and can be
downloaded from       
\texttt{\htmladdnormallink{http://www.branchandcut.org/SYMPHONY}{http://www.branchandcut.org/SYMPHONY/}}. There is also a white paper that guides the
user through the development of the matching solver.


\section{Advanced Development}
\label{development}
%===========================================================================%
%                                                                           %
% This file is part of the documentation for the SYMPHONY MILP Solver.      %
%                                                                           %
% SYMPHONY was jointly developed by Ted Ralphs (tkralphs@lehigh.edu) and    %
% Laci Ladanyi (ladanyi@us.ibm.com).                                        %
%                                                                           %
% (c) Copyright 2000-2006 Ted Ralphs. All Rights Reserved.                  %
%                                                                           %
% SYMPHONY is licensed under the Common Public License. Please see          %
% accompanying file for terms.                                              %
%                                                                           %
%===========================================================================%

\section{Orienting Yourself}

The easiest way to get oriented is to examine the organization of the source
files (note that file names will be given Unix-style). When you unpack the
\BB\ distribution, you will notice that the source files are organized along
the lines of the modules.  There is a separate directory for each
module---master ({\tt Master}), tree manager ({\tt TreeManager}), cut
generator ({\tt CutGen}), cut pool ({\tt CutPool}), and LP solver ({\tt
LP}). In addition, there is a directory called {\tt DrawGraph} and a directory
called {\tt Common} that also contain source files. The {\tt DrawGraph}
directory provides an interface from \BB\ to the {\em Interactive Graph
Drawing} software package developed by Marta Es\"o. This is an excellent
utility for graphical display and debugging. The {\tt Common} directory
contains source code for functions used by multiple modules.

Within each module's directory, there is a primary source file
containing the function {\tt main()} (named {\tt *.c} where * is the
module name), a source file containing functions related to
inter-process communication (named {\tt *\_proccomm.c}) and a file
containing general subroutines used by the module (named {\tt
*\_func.c}). The master is the exception and is organized slightly
differently. The LP process source code is further subdivided due to
the sheer number of functions.

The {\tt include} directory contains the header files. Corresponding
to each module, there are three header files, one containing internal
data structures and function prototypes associated with the module
(named {\tt *.h} where * is the module name), one containing the data
structures for storing the parameters (these are also used by the
master process), and the third containing the function prototypes for
the user callbacks (name {\tt *\_u.h}). By looking at the header files, you
should get a general idea of how things are laid out.

In addition to the subdirectories corresponding to each module, there is a
subdirectory called \texttt{SYMPHONY-\VER/SYMPHONY/Applications/USER}, which 
contains the files needed for implementing the callbacks. Before beginning
customization, it is recommended to make a copy of this directory that will 
be used as a template for creating your
customized solver. In this directory and its subdirectories, which mirror the
subdirectories of SYMPHONY itself, each file contains function stubs that can
be filled in to create a new custom application. There is one file for each
module, initially called 
\texttt{SYMPHONY-\VER/SYMPHONY/Applications/USER/*/user\_*.c}, where
\texttt{*} is the name of the module. The primary thing that you, as the user,
need to understand to build a custom application is how to fill in these
stubs. That is what the second section of this manual is about.

\section{Writing the Callbacks}

For each module, all callback functions are invoked from so-called \emph{wrapper
functions} that provide the interface and also performs a default action if
the user chooses not to override it. Although SYMPHONY is written in C, the
wrapper functions provide a C++-style interface in which the user can either
accept the default action or override it. Each wrapper function is named {\tt
*\_u()} , where {\tt *} is the name of the corresponding callback function, and is
defined in a file called {\tt *\_wrapper.c}. The wrapper function first
collects the necessary data and hands it to the user by calling the user
function. Based on the return value from the user, the wrapper then performs
any necessary post-processing. All callback functions have default options, so
that SYMPHONY now acts as a generic MILP solver out of the box.

In Section \ref{API}, the callback functions are described in
detail.  The name of every callback function starts with {\tt user\_}.
There are three kinds of arguments:
\begin{description}
\item[\rm IN:] An argument containing information that the user might need
to perform the function.
\item[\rm OUT:] A pointer to an argument in which the user should
return a result (requested data, decision, etc.) of the function. 
\item[\rm INOUT:] An argument which contains information the user might need,
but also for which the user can change the value.
\end{description}
The return values for most function are as follows:
\begin{description}
\item[Return values:] \hfill

\begin{tabular}{lp{310pt}} 

{\tt USER\_ERROR} & Error in the user function. Printing an error message is
the user's responsibility. Depending on the work the user function was
supposed to do, the error might be ignored (and some default option
used), or the process aborts. \\

{\tt USER\_SUCCESS} & The user function was implemented and executed correctly. \\

{\tt USER\_DEFAULT} & This option means that the user function was not
implemented and that SYMPHONY should either execute a default subroutine (the
default is one of the built-in options, \BB\ decides which one to use based on
initial parameter settings and the execution of the algorithm) or else do
nothing, if execution of the subroutine is optional. \\

{\tt built\_in\_option1 } & \\
{\tt built\_in\_option2 } ... & The specified built-in option will be used.\\
\end{tabular}

\item[Notes:] \hfill
\begin{itemize}
\vspace{-3ex}

\item Sometimes an output is optional. This is always noted in the
function descriptions.

\item If an array has to be returned (i.e., the argument is {\tt type
**array}) then (unless otherwise noted) the user has to allocate space for the
array itself and set {\tt *array} to be the array allocated.  If an output
array is optional and the user is not returning any values in that array, then
the user {\em must not} set {\tt *array} because this is how \BB\ decides
which optional arrays are filled up.

\item Some built-in options are implemented so that the user can invoke them
directly from the callback function. This might be useful if, for example,
the user wants to use different built-in options at different stages
of the algorithm.
\end{itemize}

\end{description}

\section{Data Structures}

\subsection{Internal Data Structures}

With few exceptions, the data structures used internally by \BB\
are undocumented and most users will not need to access them
directly. However, if such access is desired, a pointer to the main data
structure used by each of the modules can be obtained simply by calling
the function {\tt get\_*\_ptr()} where * is the appropriate module (see the
header files). This function will return a pointer to the data
structure for the appropriate module. Casual users are advised against
modifying \BB's internal data structures directly.

\subsection{User-defined Data Structures}

The user can define her own data structure for each module to maintain problem
data and any other information the user needs access to in order to implement
functions to customize the solver. A pointer to this data structure is
maintained by \BB\ and is passed to the user as an argument to each user
function. Since \BB\ knows nothing about this data structure, it is up to the
user to allocate it and maintain it. The user must also implement a function
to free it. The functions for freeing the user data structures in each module
are called \texttt{user\_free\_*}, where \texttt{*} is the module. These
functions are called by SYMPHONY at the time when other data structures for
the modules are being freed and the module is being closed. By default, for
sequential computation, there is one common user data structure for all
modules and the pointer to that data structure is passed to all user
functions, regardless of the module. This setup should work fine for most
sequential applications. In parallel, however, pointers cannot be shared
between modules and data must be explicitly passed. IN this case, it is
sometimes more efficient to maintain in each module only the data necessary to
perform the functions of that module.

\section{Inter-process Communication for Parallel Implementation}

\subsection{Distributed-memory Architectures}
\label{communication}

While the implementation of \BB\ strives to shield the user from having to
know anything about communications protocols or the specifics of inter-process
communication, it may be necessary for the user to pass information from one
module to another in order to implement a parallel application. For instance,
the user may want to pass data describing the problem instance to the LP
process after reading them in from a file in the master process. For the
purpose of passing user data from the master process to other processes, a
customization function called
\texttt{user\_send\_*\_data()} is provided in the master module, along with a
corresponding function called \texttt{user\_receive\_*\_data()} in the module
\texttt{*}. These two functions work in tandem to transport the user's data
from the maser, where it can be read in from a file, to the proper module for
processing. There are also a number of other tandem pairs of \emph{send} and
\emph{receive} functions that are used to transport user data from place to
place.

All data are sent in the form of arrays of either type {\tt char}, {\tt int},
or {\tt double}, or as strings. To send an array, the user has simply to
invoke the function {\tt send\_XXX\_array(XXX *array, int length)} where
\texttt{XXX} is one of the previously listed types. To receive that array,
there is a corresponding function called {\tt receive\_?\_array(?  *array, int
length)}. When receiving an array, the user must first allocate the
appropriate amount of memory. In cases where variable length arrays need to be
passed, the user must first pass the length of the array (as a separate array
of length one) and then the array itself. In the receive function, this allows
the length to be received first so that the proper amount of space can be
allocated before receiving the array itself. Note that data must be received
in exactly the same order as it was passed, as data is read linearly into and
out of the message buffer. The easiest way to ensure this is done properly is
to simply copy the send statements into the receive function and change the
function names. It may then be necessary to add some allocation statements in
between the receive function calls.

\subsection{Shared-memory Architectures}
\label{shared}

In the shared memory configuration, it is not necessary to use
message passing to move information from one module to another since
memory is globally accessible. In the few cases where the user would
ordinarily have to pass information using message passing, it is
easiest and most efficient to simply copy the information to the new
location. This copying gets done in the {\em send} function and hence
the {\em receive} function is never actually called. This means that
the user must perform all necessary initialization, etc. in the send
function. This makes it a little confusing to write source code which
will work for all configurations. However, the confusion should be
minimized by looking at the sample applications, especially the VRP solver,
which works in all configurations, sequential, distributed parallel, and
shared parallel. 

%FIXME: Check this
\section{Building the Application}

\subsection{Unix}

First, configure and compile SYMPHONY \VER\  as described in
Section \ref{build_appl}. This will generate the required library
and the makefiles for each application. Modify the variables in the 
SYMPHONY/Applications/USER/Makefile appropriately. 
Typing "make" in the USER subdirectory should successfully make
the USER executable. For more information, including the parallel 
configuration instuctions, see the 
SYMPHONY/Applications/USER/INSTALL file. 
After you've successfully compiled the code, you can develop your custom 
application by following the
instructions for filling in the user callback functions as described in
Section \ref{SYMPHONY-development}.

\subsection{Microsoft Windows}

First, download \texttt{SYMPHONY-\VER} and unpack the archive if it is 
required. You now have three options. You can either compile on the 
command-line using the automated MSDEV build system or NMAKE utility or you can 
use the provided projects and workspaces. Note that, in Windows, it is not 
recommended to move USER package from the SYMPHONY package. Otherwise, 
the user has to redefine all the paths to SYMPHONY and COIN header files 
as well as the paths to COIN project files. For all of the following options, 
first go to the \texttt{SYMPHONY\bs Applications\bs USER\bs Win32\bs v6} 
directory.

\subsection{Using the MSDEV Utility}
\begin{itemize}
\item Open a command line terminal and type
{\color{Brown}
\begin{verbatim}
 msdev user.dsw /make all
\end{verbatim}
}
This will create both the debug and release versions of USER application. 
If you want to compile only one of them, type
{\color{Brown}
\begin{verbatim}
 msdev user.dsw /make "all - debug"
\end{verbatim}
}
or 
{\color{Brown}
\begin{verbatim}
 msdev user.dsw /make "all - release"
\end{verbatim}
}
For each command, the executable \texttt{user} will be created in 
\texttt{Debug} and/or \texttt{Release}
directories. 

\item To test the executable, type 
{\color{Brown}
\begin{verbatim}
 Debug\user.exe -F ..\..\sample.user
\end{verbatim}
}
\item If USER source files are modified, type 
{\color{Brown}
\begin{verbatim}
 msdev user.dsw /make all /rebuild
\end{verbatim}
}
in order to clean and rebuild everything.
\end{itemize} 

\subsection{Using the NMAKE Utility}

\begin{itemize}
\item 
Edit the file \texttt{user.mak} to reflect your environment. 
Only minor edits should be required. An explanation of 
what has to be set is contained in the comments in the makefile. 
This basically 
requires the same routines that one needs to walk through in 
\texttt{SYMPHONY}'s makefile. 
See the related parts of \ref{using_nmake} section 
of \texttt{SYMPHONY} above.

\item Once configuration is done, type 
{\color{Brown}
\begin{verbatim}
  nmake /f user.mak
\end{verbatim}
}
The executable \texttt{user.exe} will be created under the 
\texttt{Debug} directory.
\item To test the executable, type 
{\color{Brown}
\begin{verbatim}
 Debug\user.exe -F ..\..\sample.user
\end{verbatim}
}
\end{itemize}

\subsection{Using the MSVC++ Workspace}

\begin{itemize}
\item Open the workspace 
\texttt{user.dsw}.

\item 
The configuration steps are exactly the same with the MSVC++ section of 
\texttt{SYMPHONY}. The only 
difference is that, you have the \texttt{user} project instead of the
\texttt{symphony} project. Go through the related steps of section 
\ref{getting_started_windows} to see how to get USER executable. 

\item
Once you have the proper settings, choose \texttt{Build
user.exe} from the \texttt{Build} menu. This should successfully 
build the executable.

\item
To test the executable, right click on the \texttt{user} project, go to the
\texttt{Debug} tab and set the program arguments to 
\texttt{-F ..\bs ..\bs sample.mps}. Note that command-line switches are 
Unix-style.

\item
Now choose \texttt{Execute} from the build menu and you have a working branch
and bound solver! After successful compilation, you can fill in the user
callback functions as describe in Section \ref{SYMPHONY-development}.
\end{itemize}

%\subsection{Unix Operating Systems}

%Once the callback functions are filled in, all that remains is to compile the
%application. The distribution comes with two makefiles that facilitate this
%process. The primary makefile resides in the {\tt SYMPHONY-\VER/} directory.
%The user makefile resides in the user's subdirectory, initially called
%\texttt{SYMPHONY-\VER/SYMPHONY/Applications/USER/}. This subdirectory can be
%moved, as well as renamed. There are a number of variables that must be set in
%the primary make file. To modify the makefiles appropriately, see the
%instructions in Section \ref{getting_started_unix}.

%\subsection{Microsoft Windows}

%First, follow the instructions for compiling SYMPHONY in Section
%\ref{getting_started_windows} to ensure you have the proper settings. Once the
%stub files in the {\tt SYMPHONY-\VER\bs SYMPHONY\bs Applications \bs USER}
%hierarchy are filled in, you should be able to compile the new application and
%run it successfully.

\section{Debugging Your Application}

Much of this section applies to Unix operating systems. However, it may
also be useful for Windows users.

\subsection{The First Rule}

\BB\ has many built-in options to make debugging easier. The most
important one, however, is the following rule. {\bf It is easier to
debug the fully sequential version than the fully distributed
version}. Debugging parallel code is not terrible, but it is more
difficult to understand what is going on when you have to look at the
interaction of several different modules running as separate
processes. This means multiple debugging windows which have to be
closed and restarted each time the application is re-run. For this
reason, it is highly recommended to develop code that can be compiled
serially even if you eventually intend to run in a fully distributed
environment. This does make the coding marginally more complex, but
believe me, it's worth the effort. The vast majority of your code will
be the same for either case. Make sure to use the configuration flag to
``{\tt -enable-debug}'' while configuring (see Section \ref{configuring}). 

\subsection{Debugging with PVM}
\label{debugging-PVM}
If you wish to venture into debugging your distributed application,
then you simply need to set the parameter {\tt *\_debug}, where * is
the name of the module you wish to debug, 
to the value ``4'' in the parameter file (the number ``4'' is chosen
by PVM). This will tell PVM to spawn the particular process or
processes in question under a debugger. What PVM actually does in this
case is to launch the script {\tt \$PVM\_ROOT/lib/debugger}. You will
undoubtedly want to modify this script to launch your preferred
debugger in the manner you deem fit. If you have trouble with this,
please send e-mail to the list serve (see Section \ref{resources}).

It's a little tricky to debug interacting parallel processes, but you
will quickly get the idea. The main difficulty is in that the order of
operations is difficult to control. Random interactions can occur when
processes run in parallel due to varying system loads, process
priorities, etc. Therefore, it may not always be possible to duplicate
errors. To force runs that you should be able to reproduce, make sure
the parameter \ptt{no\_cut\_timeout} appears in the parameter file or
start \BB\ with the ``{\tt -a}'' option. This will keep the cut
generator from timing out, a major source of randomness. Furthermore,
run with only one active node allowed at a time (set \ptt{
max\_active\_nodes} to ``1''). This will keep the tree search from
becoming random. These two steps should allow runs to be reproduced.
You still have to be careful, but this should make things easier.

\subsection{Using {\tt Purify} and {\tt Quantify}}

The makefile is already set up for compiling applications using {\tt
purify} and {\tt quantify}. Simply set the paths to the executables
and type ``{\tt make pall}'' or ``{\tt p*}'' where * is the module you
want to purify. The executable name is the same as described in
Section \ref{distributed-build}, but with a ``p'' in front of it. To tell PVM
to launch the purified version of the executables, you must set the
parameters {\tt *\_exe} in the parameter file to the purified
executable names. See Section \ref{tm_params} for information on
setting parameters in the parameter file.

\subsection{Checking the Validity of Cuts and Tracing the Optimal Path}
\label{debugging}
Sometimes the only evidence of a bug is the fact that the optimal
solution to a particular problem is never found. This is usually
caused by either (1) adding an invalid cut, or (2) performing an
invalid branching. There are two options available for discovering
such errors. The first is for checking the validity of added cuts.
This checking must, of course, be done by the user, but \BB\ can
facilitate such checking. To do this, the user must fill in the
function \hyperref{{\tt user\_check\_validity\_of\_cut()}} {\ptt{
user\_check\_validity\_of\_cut()} (see Section
}{)}{user_check_validity_of_cut}. THIS function is called every time a
cut is passed from the cut generator to the LP and can function as an
independent verifier. To do this, the user must pass (through her own
data structures) a known feasible solution. Then for each cut passed
into the function, the user can check whether the cut is satisfied
by the feasible solution. If not, then there is a problem! Of course,
the problem could also be with the checking routine. 
After filling in this function, the user must configure SYMPHONY
with the flag \texttt{--enable-cut-check} (see Section \ref{configuring}). 
Type ``{\tt make clean}'' and then ``{\tt make}.''

Tracing the optimal path can alert the user when the subproblem which
admits a particular known feasible solution (at least
according to the branching restrictions that have been imposed so far)
is pruned. This could be due to an invalid branching. Note that this
option currently only works for branching on binary variables. To use
this facility, the user must fill in the function \hyperref{{\tt
user\_send\_feas\_sol()}} {\ptt {user\_send\_feas\_sol()} (see Section
}{)}{user_send_feas_sol}. All that is required is to pass out an array
of user indices that are in the feasible solution that you want to
trace. Each time the subproblem which admits this feasible solution is
branched on, the branch that continues to admit the solution is
marked. When one of these marked subproblems is pruned, the user is
notified.

\subsection{Using the {\tt Interactive Graph Drawing} Software}
\label{IGD}
The Interactive Graph Drawing (IGD) software package is included with
\BB\ and \BB\ facilitates its use through interfaces with the
package. The package, which is a Tcl/Tk application, is extremely
useful for developing and debugging applications involving graph-based
problems. Given display coordinates for each node in the graph, IGD
can display support graphs corresponding to fractional solutions with or
without edge weights and node labels and weights, as well as other
information. Furthermore, the user can interactively modify the graph
by, for instance, moving the nodes apart to ``disentangle'' the
edges. The user can also interactively enter violated cuts through the
IGD interface.

To use IGD, you must have installed PVM since the drawing window runs
as a separate application and communicates with the user's routines
through message passing. To compile the graph drawing application,
type ``{\tt make dg}'' in the \BB\ root directory. The user
routines in the file {\tt user\_dg.c} can be filled in, but it is not
necessary to fill anything in for basic applications. 

After compiling {\tt dg}, the user must write some subroutines that
communicate with {\tt dg} and cause the graph to be drawn.
Regrettably, this is currently a little more complicated than it needs
to be and is not well documented. However, by looking at the sample
application, it should be possible to see how it is done. To
enable graph drawing, put the line {\ptt {do\_draw\_graph 1} into the
parameter file or use the {\tt -d} command line option. It can be difficult to
get IGD to work. If you are interested in using it and cannot get it to work,
feel free to contact me.

\subsection{Other Debugging Techniques}

Another useful built-in function is \texttt{write\_mps()}, which will write the
current LP relaxation to a file in MPS format. This file can then be read into
the LP solver interactively or examined by hand for errors.  Many times, CPLEX
gives much more explicit error messages interactively than through the
callable library. The form of the function is
\begin{verbatim}
void write_mps(LPdata *lp_data, char *fname)
\end{verbatim}
where \texttt{fname} is the name of the file to be written. If \BB\ is forced
to abandon solution of an LP because the LP solver returns an error code, the
current LP relaxation is automatically written to the file ``{\tt
matrix.[bc\_index].[iter\_num].mps}'' where {\em bc\_index} is the index of
the current subproblem and {\em iter\_num} is the current iteration
number. The \texttt{write\_mps()} function can be called using breakpoint code
to examine the status of the matrix at any point during execution.

Logging is another useful feature. Logging the state of the search tree can
help isolate some problems more easily. See Section \ref{tm_params}
for the appropriate parameter settings to use logging.

\section{Controlling Execution and Output}
\label{output}
Calling \BB\ with no arguments simply lists all command-line options.  Most of
the common parameters can be set on the command line. Sometimes, however, it
may be easier to use a parameter file. To invoke \BB\ with a parameter file
type ``{\tt master -f filename ...}'' where filename is the name of the
parameter file. The format of the file is explained in Section
\ref{parameter_file}. 

The output level can be controlled through the use of the verbosity
parameter. Setting this parameter at different levels will cause
different progress messages to be printed out. Level 0 only prints out
the introductory and solution summary messages, along with status
messages every 10 minutes. Level 1 prints out a message every time a
new node is created. Level 3 prints out messages describing each
iteration of the solution process. Levels beyond 3 print out even more
detailed information.

There are also two possible graphical interfaces. For graph-based
problems, the Interactive Graph Drawing Software allows visual display
of fractional solutions, as well as feasible and optimal solutions
discovered during the solution process. For all types of problems,
VBCTOOL creates a visual picture of the branch and cut tree, either
in real time as the solution process evolves or as an emulation from a
file created by
\BB. See Section \ref{tm_params} for information on how to use VBCTOOL
with SYMPHONY. Binaries for VBCTOOL can be obtained at \\ 
{\tt \htmladdnormallink
{http://www.informatik.uni-koeln.de/ls\_juenger/projects/vbctool.html}
{http://www.informatik.uni-koeln.de/ls\_juenger/projects/vbctool.html}}.





%\newpage

%\section{Data Structures}

%\input{man-data-structures}

\chapter{Reference}

\section{The User API Specification}
\label{API}
\sloppy
%%%%%%%%%%%%%%%%%%%%%%%%%%%%%%%%%%%%%%%%%%%%%%%%%%%%%%%%%%%%%%%%%%%%%%%%%%%%%

\subsection{User-written functions of the Master process}

%begin{latexonly}
\bd
%end{latexonly}

%%%%%%%%%%%%%%%%%%%%%%%%%%%%%%%%%%%%%%%%%%%%%%%%%%%%%%%%%%%%%%%%%%%%%%%%%%%%%
% user_usage
%%%%%%%%%%%%%%%%%%%%%%%%%%%%%%%%%%%%%%%%%%%%%%%%%%%%%%%%%%%%%%%%%%%%%%%%%%%%%

\functiondef{user\_usage}
\label{user_usage}
\begin{verbatim}
void user_usage()
\end{verbatim}

\bd

\describe

The user can use any capitol letter (except 'H') for command line
switches to control user-defined parameter settings without the use of
a parameter file. The function {\tt user\_usage()} can optionally
print out usage information for the user-defined command line
switches. The command line switch {\tt -H} automatically calls the
user's usage subroutine. The switch {\tt -h} prints
\BB's own usage information.

\ed

\vspace{1ex}
%%%%%%%%%%%%%%%%%%%%%%%%%%%%%%%%%%%%%%%%%%%%%%%%%%%%%%%%%%%%%%%%%%%%%%%%%%%%%
% user_initialize
%%%%%%%%%%%%%%%%%%%%%%%%%%%%%%%%%%%%%%%%%%%%%%%%%%%%%%%%%%%%%%%%%%%%%%%%%%%%%

\functiondef{user\_initialize}
\begin{verbatim}
int user_initialize(void **user)
\end{verbatim}

\bd

\describe

The user allocates space for and initializes the user-defined
data structures for the master process.

\args

\bt{llp{250pt}}
{\tt void **user} & OUT & Pointer to the user-defined data structure. \\
\et

\returns

\bt{lp{300pt}}
{\tt ERROR} & Error. \BB\ stops. \\
{\tt USER\_NO\_PP} & Initialization is done. \\
\et

\ed

\vspace{1ex}

%%%%%%%%%%%%%%%%%%%%%%%%%%%%%%%%%%%%%%%%%%%%%%%%%%%%%%%%%%%%%%%%%%%%%%%%%%%%%
% user_free_master
%%%%%%%%%%%%%%%%%%%%%%%%%%%%%%%%%%%%%%%%%%%%%%%%%%%%%%%%%%%%%%%%%%%%%%%%%%%%%

\functiondef{user\_free\_master}
\begin{verbatim}
int user_free_master(void **user)
\end{verbatim}

\bd

\describe

The user frees all the data structures within {\tt *user}, and
also free {\tt *user} itself. This can be done using the built-in macro
{\tt FREE} that checks the existence of a pointer before freeing it.

\args

\bt{llp{280pt}}
{\tt void **user} & INOUT & Pointer to the user-defined data structure
(should be {\tt NULL} on return). \\
\et

\returns

\bt{lp{300pt}}
{\tt ERROR} & Ignored. This is probably not a fatal error.\\
{\tt USER\_NO\_PP} & Everything was freed successfully. \\
\et

\ed

\vspace{1ex}

%%%%%%%%%%%%%%%%%%%%%%%%%%%%%%%%%%%%%%%%%%%%%%%%%%%%%%%%%%%%%%%%%%%%%%%%%%%%%
% user_readparams
%%%%%%%%%%%%%%%%%%%%%%%%%%%%%%%%%%%%%%%%%%%%%%%%%%%%%%%%%%%%%%%%%%%%%%%%%%%%%

\functiondef{user\_readparams}
\begin{verbatim}
int user_readparams(void *user, char *filename, int argc, char **argv)
\end{verbatim}

\bd

\describe

The user reads in parameters from the file named {\tt filename}. The
file {\tt filename} is a file containing both built-in parameters and
user parameters. The filename is given as a command line argument when
starting the application and is then passed to the user. The user must
open the file for reading, scan the file for lines that contain user
parameters and then read the parameters in as appropriate. See the
file {\tt Master/master\_io.c} to see how \BB\ does this.

Optionally, the user can also parse the command line arguments. All
capital letters are reserved for user-defined command line switches.
The switch {\tt -H} is reserved for help and calls the user's usage
subroutine (see {\tt user\_send\_lp\_data()}).

\args

\bt{llp{280pt}}
{\tt void *user} & IN & Pointer to the user-defined data structure. \\
{\tt char *filename} & IN & The name of the parameter file. \\
\et

\returns

\bt{lp{300pt}}
{\tt ERROR} & Error. \BB\ stops. \\
{\tt USER\_NO\_PP} & User parameters were read successfully. \\
\et

\ed

\vspace{1ex}

%%%%%%%%%%%%%%%%%%%%%%%%%%%%%%%%%%%%%%%%%%%%%%%%%%%%%%%%%%%%%%%%%%%%%%%%%%%%%
% user_io
%%%%%%%%%%%%%%%%%%%%%%%%%%%%%%%%%%%%%%%%%%%%%%%%%%%%%%%%%%%%%%%%%%%%%%%%%%%%%

\functiondef{user\_io}
\begin{verbatim}
int user_io(void *user)
\end{verbatim}

\bd

\describe

The user prepares all information needed to specify the problem
instance (e.g., reads in data from a data file, etc.).

\args

\bt{llp{280pt}}
{\tt void *user} & IN & Pointer to the user-defined data structure. \\
\et

\returns

\bt{lp{300pt}}
{\tt ERROR} & Error. \BB\ stops. \\
{\tt USER\_NO\_PP} & User I/O was completed successfully. \\
\et

\ed

\vspace{1ex}

%%%%%%%%%%%%%%%%%%%%%%%%%%%%%%%%%%%%%%%%%%%%%%%%%%%%%%%%%%%%%%%%%%%%%%%%%%%%%
% user_init_draw_graph
%%%%%%%%%%%%%%%%%%%%%%%%%%%%%%%%%%%%%%%%%%%%%%%%%%%%%%%%%%%%%%%%%%%%%%%%%%%%%

\functiondef{user\_init\_draw\_graph}
\begin{verbatim}
int user_init_draw_graph(void *user, int dg_id)
\end{verbatim}

\bd

\describe

This function is invoked only if the {\tt do\_draw\_graph} parameter is set.
The user can initialize the graph drawing process by sending
some initial information (e.g., the location of the nodes of a
graph, like in the TSP.)

\args

\bt{llp{280pt}}
{\tt void *user} & IN & Pointer to the user-defined data structure. \\
{\tt int dg\_id} & IN & The process id of the graph drawing process. \\
\et

\returns

\bt{lp{300pt}}
{\tt ERROR} & Error. \BB\ stops. \\
{\tt USER\_NO\_PP} & The user completed initialization successfully. \\
\et

\ed

\vspace{1ex}

%%%%%%%%%%%%%%%%%%%%%%%%%%%%%%%%%%%%%%%%%%%%%%%%%%%%%%%%%%%%%%%%%%%%%%%%%%%%%
% user_start_heurs
%%%%%%%%%%%%%%%%%%%%%%%%%%%%%%%%%%%%%%%%%%%%%%%%%%%%%%%%%%%%%%%%%%%%%%%%%%%%%

\functiondef{user\_start\_heurs}
\label{user_start_heurs}
\begin{verbatim}
int user_start_heurs(void *user, double *ub, double *ub_estimate)
\end{verbatim}

\bd

\describe

The user invokes heuristics and generates the initial global upper
bound and also perhaps an upper bound estimate. This is the last place 
where the user can do things before the
branch and cut algorithm starts. She might do some preprocessing,
in addition to generating the upper bound.

\args

\bt{llp{300pt}}
{\tt void *user} & IN & Pointer to the user-defined data structure. \\
{\tt double *ub} & OUT & Pointer to the global upper bound. Initially,
the upper bound is set to either {\tt -MAXDOUBLE} or the bound read in
from the parameter file, and should be changed by the user only if
a better valid upper bound is found. \\
{\tt double *ub\_estimate} & OUT & Pointer to an estimate of the global
upper bound. This is useful if the {\tt BEST\_ESTIMATE} diving
strategy is used (see the treemanager parameter
\hyperref{\tt diving\_strategy}{{\tt diving\_strategy} (Section } {)}
{diving_strategy}) \\
\et

\returns

\bt{lp{300pt}}
{\tt ERROR} & Error. This error is probably not fatal. \\
{\tt USER\_NO\_PP} & User executed function successfully. \\
\et

\ed

\vspace{1ex}

%%%%%%%%%%%%%%%%%%%%%%%%%%%%%%%%%%%%%%%%%%%%%%%%%%%%%%%%%%%%%%%%%%%%%%%%%%%%%
% user_set_base
%%%%%%%%%%%%%%%%%%%%%%%%%%%%%%%%%%%%%%%%%%%%%%%%%%%%%%%%%%%%%%%%%%%%%%%%%%%%%

\functiondef{user\_set\_base}
\begin{verbatim}
int user_set_base(void *user, int *basevarnum, int **basevars, double **lb,
                  double **ub, int *basecutnum, int *colgen_strat)
\end{verbatim}

\bd

\describe

The user must specify the set of base variables and the number of base
constraints. The base constraints themselves need not be specified
since they are never stored explicitly.

\args

\bt{llp{245pt}}
{\tt void *user} & IN & Pointer to the user-defined data structure. \\
{\tt int *varnum} & OUT & Pointer to the number of base variables. \\
{\tt int **userind} & OUT & Pointer to an array containing the user
indices of the base variables. \\
{\tt int **lb} & OUT & Pointer to an array containing the lower bounds for
the base variables. \\
{\tt int **ub} & OUT & Pointer to an array containing the upper bounds for
the base variables. \\
{\tt int *cutnum} & OUT & The number of base constraints. \\
{\tt int *colgen\_strat} & INOUT & The default strategy or one that has
been read in from the parameter file is passed in, but the user is free
to change it. See {\tt colgen\_strat} in the description of
parameters for details on how to set it.
\et

\returns

\bt{lp{300pt}}
{\tt ERROR} & Error. \BB\ stops. \\
{\tt USER\_NO\_PP} & The required data are filled in, but no
post-processing done. \\
{\tt USER\_AND\_PP} & All required post-processing done. \\
\et

\postp

The array of user indices is sorted if the user has not already done so.

\ed

\vspace{1ex}

%%%%%%%%%%%%%%%%%%%%%%%%%%%%%%%%%%%%%%%%%%%%%%%%%%%%%%%%%%%%%%%%%%%%%%%%%%%%%
% user_create_root
%%%%%%%%%%%%%%%%%%%%%%%%%%%%%%%%%%%%%%%%%%%%%%%%%%%%%%%%%%%%%%%%%%%%%%%%%%%%%

\functiondef{user\_create\_root}
\begin{verbatim}
int user_create_root(void *user, int *extravarnum, int **extravars)
\end{verbatim}

\bd

\describe

The user must specify which extra variables are to be active in the root
node in addition to the base variables.

\args

\bt{llp{265pt}}
{\tt void *user} & IN & Pointer to the user-defined data structure. \\
{\tt int *extravarnum} & OUT & Pointer to the number of extra active
variables in the root. \\
{\tt int *extravars} & OUT & Pointer to an array containing a list of
user indices of the extra variables to be active in the root. \\
\et

\returns

\bt{lp{300pt}}
{\tt ERROR} & Error. \BB\ stops. \\
{\tt USER\_NO\_PP} & All required data  filled out, but no
post-processing done. \\
{\tt USER\_AND\_PP} & All required post-processing done. \\
\et

\postp

The array of extra indices is sorted if the user has not already done so.

\ed

\vspace{1ex}

%%%%%%%%%%%%%%%%%%%%%%%%%%%%%%%%%%%%%%%%%%%%%%%%%%%%%%%%%%%%%%%%%%%%%%%%%%%%%
% user_receive_feasible_solution
%%%%%%%%%%%%%%%%%%%%%%%%%%%%%%%%%%%%%%%%%%%%%%%%%%%%%%%%%%%%%%%%%%%%%%%%%%%%%

\functiondef{user\_receive\_feasible\_solution}
\label{user_receive_feasible_solution}
\begin{verbatim}
int user_receive_feasible_solution(void *user, int msgtag, double cost, 
                                   int numvars, int *indices, double *values)
\end{verbatim}

\bd

\describe

Feasible solutions can be sent and/or stored in a user-defined packed
form if desired. For instance, the TSP, a tour can be specified simply
as a permutation, rather than as a list of variable indices. In the LP
process, a feasible solution is packed either by the user or by a
default packing routine. If the default packing routine was used, the
{\tt msgtag} will be {\tt FEASIBLE\_SOLUTION\_NONZEROS}. In this case,
{\tt cost}, {\tt numvars}, {\tt indices} and {\tt values} will contain
the solution value, the number of nonzeros in the feasible solution,
and their user indices and values. The user has only to interpret and
store the solution. Otherwise, when {\tt msgtag} is {\tt
FEASIBLE\_SOLUTION\_USER}, \BB\ will send and receive the solution
value only and the user has to unpack exactly what she has packed in
the LP process. In this case the contents of the last three arguments
are undefined.

\args

\bt{llp{290pt}}
{\tt void *user} & IN & Pointer to the user-defined data structure. \\
{\tt int msgtag} &    IN & {\tt FEASIBLE\_SOLUTION\_NONZEROS} or {\tt
FEASIBLE\_SOLUTION\_USER} \\
{\tt double cost}  &    IN & The cost of the feasible solution.\\
{\tt int numvars} &  IN & The number of variables whose user indices and
values were sent (length of {\tt indices} and {\tt values}). \\
{\tt int *indices} &  IN & The user indices of the nonzero variables. \\
{\tt double *values} & IN & The corresponding values. \\
\et

\returns

\bt{lp{300pt}}
{\tt ERROR} & Ignored. This is probably not a fatal error.\\
{\tt USER\_NO\_PP} & The solution has been unpacked and stored. \\
\et

\ed

\vspace{1ex}

%%%%%%%%%%%%%%%%%%%%%%%%%%%%%%%%%%%%%%%%%%%%%%%%%%%%%%%%%%%%%%%%%%%%%%%%%%%%%
% user_send_lp_data
%%%%%%%%%%%%%%%%%%%%%%%%%%%%%%%%%%%%%%%%%%%%%%%%%%%%%%%%%%%%%%%%%%%%%%%%%%%%%

\functiondef{user\_send\_lp\_data}
\label{user_send_lp_data}
\begin{verbatim}
int user_send_lp_data(void *user, void **user_lp)
\end{verbatim}

\bd

\describe

The user has to send all problem-specific data that will be needed in
the LP process to set up the initial LP relaxation and perform later
computations. This could include instance data, as well as user
parameter settings. This is one of the few places where the user will
need to worry about the configuration of the modules. If either the
tree manager or the LP are running as a separate process (either {\tt
COMPILE\_IN\_LP} or {\tt COMPILE\_IN\_TM} are {\tt FALSE} in the {\tt
make} file), then the data will be sent and received through
message-passing. See \hyperref{{\tt user\_receive\_lp\_data()}}
{{\tt user\_receive\_lp\_data()} in Section }{}{user_receive_lp_data} for more
discussion. Otherwise, it can be copied over directly to the
user-defined data structure for the LP. In the latter case, {\tt
*user\_lp} is a pointer to the user-defined data structure for the LP
that must be allocated and initialized. For a discussion of
message-passing in \BB, see Section \ref{communication}. The code
for the two cases is put in the same source file by use of {\tt
\#ifdef} statements. See the comments in the code stub for this
function for more details.

\args

\bt{llp{275pt}}
{\tt void *user} & IN & Pointer to the user-defined data structure. \\
{\tt void **user\_lp} & OUT & Pointer to the user-defined data
structure for the LP process. \\
\et

\returns

\bt{lp{300pt}}
{\tt ERROR} & Error. \BB\ stops. \\
{\tt USER\_NO\_PP} & Packing is done. \\
\et

\ed

\vspace{1ex}

%%%%%%%%%%%%%%%%%%%%%%%%%%%%%%%%%%%%%%%%%%%%%%%%%%%%%%%%%%%%%%%%%%%%%%%%%%%%%
% user_send_cg_data
%%%%%%%%%%%%%%%%%%%%%%%%%%%%%%%%%%%%%%%%%%%%%%%%%%%%%%%%%%%%%%%%%%%%%%%%%%%%%

\functiondef{user\_send\_cg\_data}
\label{user_send_cg_data}
\begin{verbatim}
int user_pack_cg_data(void *user, void **user_cg)
\end{verbatim}

\bd

\describe

The user has to send all problem-specific data that will be needed by
the cut generator for separation. This is one of the few places where
the user will need to worry about the configuration of the modules. If
either the tree manager, the LP, or the cut generator are running as a
separate process (either {\tt COMPILE\_IN\_LP}, {\tt COMPILE\_IN\_TM},
or {\tt COMPILE\_IN\_CG} are {\tt FALSE} in the make file), then
the data will be sent and received through message-passing. See 
\hyperref{{\tt user\_receive\_cg\_data()}}
{{\tt user\_receive\_cg\_data} in Section }{}{user_receive_cg_data} 
for more discussion. Otherwise, it can be
copied over directly to the user-defined data structure for the CG. In
the latter case, {\tt *user\_cg} is a pointer to the user-defined data
structure for the CG that must be allocated and initialized. For a
discussion of message-passing in \BB, see Section
\ref{communication}. The code for the two cases is put in the same
source file by use of {\tt \#ifdef} statements. See the comments in
the code stub for this function for more details.

\args

\bt{llp{275pt}}
{\tt void *user} & IN & Pointer to the user-defined data structure. \\
{\tt void **user\_cg} & OUT & Pointer to the user-defined data
structure for the cut generator process. \\
\et

\returns

\bt{lp{300pt}}
{\tt ERROR} & Error. \BB\ stops. \\
{\tt USER\_NO\_PP} & Packing is done. \\
\et

\ed

\vspace{1ex}

%%%%%%%%%%%%%%%%%%%%%%%%%%%%%%%%%%%%%%%%%%%%%%%%%%%%%%%%%%%%%%%%%%%%%%%%%%%%%
% user_send_cp_data
%%%%%%%%%%%%%%%%%%%%%%%%%%%%%%%%%%%%%%%%%%%%%%%%%%%%%%%%%%%%%%%%%%%%%%%%%%%%%

\functiondef{user\_send\_cp\_data}
\label{user_send_cp_data}
\begin{verbatim}
int user_pack_cp_data(void *user, void **user_cp)
\end{verbatim}

\bd

\describe

The user has to send all problem-specific data that will be needed by
the cut pool in order to store and check cuts. This is one of the few
places where the user will need to worry about the configuration of
the modules. If either the tree manager, the LP, or the cut pool are
running as a separate process(either {\tt COMPILE\_IN\_LP}, {\tt
COMPILE\_IN\_TM}, or {\tt COMPILE\_IN\_CP} are {\tt FALSE} in the {\tt
make} file), then the data will be sent and received through
message-passing. See \hyperref{{\tt user\_receive\_cp\_data()}} {{\tt
user\_receive\_cp\_data()} in Section }{}{user_receive_cp_data} for more
discussion. Otherwise, it can be copied over directly to the
user-defined data structure for the CP. In the latter case, {\tt
*user\_cp} is a pointer to the user-defined data structure for the CP
that must be allocated and initialized. For a discussion of message
passing in \BB, see Section \ref{communication}. The code for the two
cases is put in the same source file by use of {\tt \#ifdef}
statements. See the comments in the code stub for this function for
more details.

\args

\bt{llp{275pt}}
{\tt void *user} & IN & Pointer to the user-defined data structure. \\
{\tt void **user\_cp} & OUT & Pointer to the user-defined data
structure for the cut pool process. \\
\et

\returns

\bt{lp{300pt}}
{\tt ERROR} & Error. \BB\ stops. \\
{\tt USER\_NO\_PP} & Packing is done. \\
\et

\ed

\vspace{1ex}

%%%%%%%%%%%%%%%%%%%%%%%%%%%%%%%%%%%%%%%%%%%%%%%%%%%%%%%%%%%%%%%%%%%%%%%%%%%%%
% user_display_solution
%%%%%%%%%%%%%%%%%%%%%%%%%%%%%%%%%%%%%%%%%%%%%%%%%%%%%%%%%%%%%%%%%%%%%%%%%%%%%

\functiondef{user\_display\_solution}
\label{user_display_solution}
\begin{verbatim}
int user_display_solution(void *user)
\end{verbatim}

\bd

\describe

This function is invoked when the best solution found so far is to be
displayed (after heuristics, after the end of the first phase, or the end of
the whole algorithm). This can be done using either a text-based
format or using the {\tt drawgraph} process.

\returns

\bt{lp{300pt}}
{\tt ERROR} & Ignored. \\
{\tt USER\_NO\_PP} & Displaying is done. \\
\et

\args

\bt{llp{280pt}}
{\tt void *user} & IN & Pointer to the user-defined data structure. \\
\et

\ed

\vspace{1ex}

%%%%%%%%%%%%%%%%%%%%%%%%%%%%%%%%%%%%%%%%%%%%%%%%%%%%%%%%%%%%%%%%%%%%%%%%%%%%%
% user_send_feas_sol
%%%%%%%%%%%%%%%%%%%%%%%%%%%%%%%%%%%%%%%%%%%%%%%%%%%%%%%%%%%%%%%%%%%%%%%%%%%%%

\functiondef{user\_send\_feas\_sol}
\label{user_send_feas_sol}
\begin{verbatim}
int user_process_own_messages(void *user, int *feas_sol_size, int **feas_sol)
\end{verbatim}

\bd

\describe

This function is useful for debugging purposes. It passes a known
feasible solution to the tree manager. The tree manager then tracks
which current subproblem admits this feasible solution and notifies
the user when it gets pruned. It is useful for finding out why a known
optimal solution never gets discovered. Usually, this is due to either
an invalid cut of an invalid branching. Note that this feature only
works when branching on binary variables. See Section \ref{debugging}
for more on how to use this feature.

\returns

\args

\bt{llp{245pt}}
{\tt void *user} & IN & Pointer to the user-defined data structure. \\
{\tt int *feas\_sol\_size} & INOUT & Pointer to size of the feasible
solution passed by the user. \\
{\tt int **feas\_sol} & INOUT & Pointer to the array of user indices
containing the feasible solution. This array is simply copied by the tree
manager and must be freed by the user. \\
\et

\bt{lp{260pt}}
{\tt ERROR} & Solution tracing is not enabled. \\
{\tt USER\_NO\_PP} & Tracing of the given solution is enabled. \\
\et

\ed

\vspace{1ex}
%%%%%%%%%%%%%%%%%%%%%%%%%%%%%%%%%%%%%%%%%%%%%%%%%%%%%%%%%%%%%%%%%%%%%%%%%%%%%
% user_process_own_messages
%%%%%%%%%%%%%%%%%%%%%%%%%%%%%%%%%%%%%%%%%%%%%%%%%%%%%%%%%%%%%%%%%%%%%%%%%%%%%

\functiondef{user\_process\_own\_messages}
\begin{verbatim}
int user_process_own_messages(void *user, int msgtag)
\end{verbatim}

\bd

\describe

The user must receive any message he sends to the master process
(independently of \BB's own messages). An example for such a message is
sending feasible solutions from separate heuristics processes fired up
in \htmlref{\tt user\_start\_heurs()}{user_start_heurs}. 

\args

\bt{llp{280pt}}
{\tt void *user} & IN & Pointer to the user-defined data structure. \\
{\tt int msgtag} & IN & The message tag of the message. \\
\et

\returns

\bt{lp{300pt}}
{\tt ERROR} & Ignored. \\
{\tt USER\_NO\_PP} & Message is processed. \\
\et

\ed

\vspace{1ex}

%begin{latexonly}
\ed
%end{latexonly}


\newpage

\input{man-lp}

\newpage

%%%%%%%%%%%%%%%%%%%%%%%%%%%%%%%%%%%%%%%%%%%%%%%%%%%%%%%%%%%%%%%%%%%%%%%%%%%%%
\subsection{User-written functions of the CG module}

Due to the relative simplicity of the cut generator, there are no wrapper
functions implemented for CG. Consequently, there are no default
options and no post-processing.

%begin{latexonly}
\bd
%end{latexonly}

%%%%%%%%%%%%%%%%%%%%%%%%%%%%%%%%%%%%%%%%%%%%%%%%%%%%%%%%%%%%%%%%%%%%%%%%%%%%%
% user_receive_cg_data
%%%%%%%%%%%%%%%%%%%%%%%%%%%%%%%%%%%%%%%%%%%%%%%%%%%%%%%%%%%%%%%%%%%%%%%%%%%%%
\functiondef{user\_receive\_cg\_data}
\label{user_receive_cg_data}
\begin{verbatim}
int user_receive_cg_data (void **user)
\end{verbatim}

\bd

\describe

The user has to receive here all problem-specific information that is
known to the master and will be needed for computation in the CG
module later on. The same data must be received here that was sent in
the \hyperref{{\tt user\_send\_cg\_data()}}{{\tt
user\_send\_cg\_data()} (see Section }{)}{user_send_cg_data} function
in the master module. The user has to allocate space for all the data
structures, including {\tt user} itself. Note that some or all of this
may be done in the function {\tt 
\htmlref{user\_send\_cg\_data()}{user_send_cg_data}} if the Tree Manager, LP,
and CG are all compiled together. See that function for more
information.

\args

\bt{llp{250pt}}
{\tt void **user} & INOUT & Pointer to the user-defined data structure. \\
\et

\returns

\bt{lp{300pt}}
{\tt ERROR} & Error. CG exits. \\
{\tt USER\_NO\_PP} & The user received the data properly. \\
\et

\item[Invoked from:] {\tt cg\_initialize()} at process start.

\ed

\vspace{1ex}

%%%%%%%%%%%%%%%%%%%%%%%%%%%%%%%%%%%%%%%%%%%%%%%%%%%%%%%%%%%%%%%%%%%%%%%%%%%%%
% user_unpack_lp_solution_cg
%%%%%%%%%%%%%%%%%%%%%%%%%%%%%%%%%%%%%%%%%%%%%%%%%%%%%%%%%%%%%%%%%%%%%%%%%%%%%

\functiondef{user\_receive\_lp\_solution\_cg}
\begin{verbatim}
int user_receive_lp_solution_cg(void *user)
\end{verbatim}

\bd

\describe

This function is invoked only if in the {\tt
\htmlref{user\_send\_lp\_solution()}{user_send_lp_solution}} function
of the LP module the user opted for packing the current LP solution
himself. Here he must unpack the very same data he packed there.

\args

\bt{llp{250pt}}
{\tt void *user} & IN & Pointer to the user-defined data structure. \\
\et

\item[Invoked from:] Whenever an LP solution is received.

\returns

\bt{lp{300pt}}
{\tt ERROR} & Error. This LP solution is not processed. \\
{\tt USER\_NO\_PP} & The user received the LP solution. \\
\et

\item[Note:] \hfill

\BB\ automatically unpacks the level, index and iteration number
corresponding to the current LP solution within the current search tree node
as well as the objective value and upper bound.

\ed

\vspace{1ex}

%%%%%%%%%%%%%%%%%%%%%%%%%%%%%%%%%%%%%%%%%%%%%%%%%%%%%%%%%%%%%%%%%%%%%%%%%%%%%
% user_free_cg
%%%%%%%%%%%%%%%%%%%%%%%%%%%%%%%%%%%%%%%%%%%%%%%%%%%%%%%%%%%%%%%%%%%%%%%%%%%%%

\functiondef{user\_free\_cg}
\begin{verbatim}
int user_free_cg(void **user)
\end{verbatim}

\bd

\describe

The user has to free all the data structures within {\tt user}, and also free
{\tt user} itself. The user can use the built-in macro {\tt FREE} that checks
the existence of a pointer before freeing it. 

\args

\bt{llp{280pt}}
{\tt void **user} & INOUT & Pointer to the user-defined data structure
(should be {\tt NULL} on exit from this function). \\
\et

\returns

\bt{lp{300pt}}
{\tt ERROR} & Ignored. \\
{\tt USER\_NO\_PP} & The user freed all data structures. \\
\et

\item[Invoked from:] {\tt cg\_close()} at module shutdown. 

\ed

\vspace{1ex}

%%%%%%%%%%%%%%%%%%%%%%%%%%%%%%%%%%%%%%%%%%%%%%%%%%%%%%%%%%%%%%%%%%%%%%%%%%%%%
% user_find_cuts
%%%%%%%%%%%%%%%%%%%%%%%%%%%%%%%%%%%%%%%%%%%%%%%%%%%%%%%%%%%%%%%%%%%%%%%%%%%%%

\functiondef{user\_find\_cuts}
\begin{verbatim}
int user_find_cuts(void *user, int varnum, int iter_num, int level,
		    int index, double objval, int *indices, double *values,
		    double ub, double lpetol, int *cutnum)
\end{verbatim}

\bd

\describe

The user can generate cuts based on the current LP solution stored in
{\tt soln}. Cuts found need to be sent back to the LP by calling the
{\tt cg\_send\_cut(\htmlref{cut\_data}{cut_data} *new\_cut)} function.
The argument of this function is a pointer to the cut to be sent. See
Section
\ref{user-written-lp} for a description of this data structure. If the
user wants the cut to be added to the cut pool in case it proves to be
effective in the LP, then {\tt new\_cut->name} should be set to {\tt
CUT\_\_SEND\_TO\_CP}. Otherwise, it should be set to {\tt
CUT\_\_DO\_NOT\_SEND\_TO\_CP}.\\
\\
The only output of this function is the number of cuts generated and this
value is returned in the last argument.

\args

\bt{llp{280pt}}
{\tt void *user} & IN & Pointer to the user-defined data structure.
\\
{\tt int iter\_num} & IN & The iteration number of the current LP solution. \\
{\tt int level} & IN & The level in the tree on which the current LP
solution was generated. \\
{\tt index} & IN & The index of the node in which LP solution was generated.
\\
{\tt objval} & IN & The objective function value of the current LP solution.
\\
{\tt int varnum} & IN & The number of nonzeros in the current LP solution. \\
{\tt indices} & IN & The column indices of the nonzero variables in the current
LP solution. \\
{\tt values} & IN & The values of the nonzero variables listed in 
{\tt indices}.
\\
{\tt double ub} & IN & The current global upper bound. \\
{\tt double lpetol} & IN & The current error tolerance in the LP. \\
{\tt int *cutnum} & OUT & Pointer to the number of cuts generated
and sent to the LP. \\
\et

\returns

\bt{lp{300pt}}
{\tt ERROR} & Ignored. \\
{\tt USER\_NO\_PP} & The user function exited properly. \\
\et

\item[Invoked from:] Whenever an LP solution is received.

\ed

\vspace{1ex}

%%%%%%%%%%%%%%%%%%%%%%%%%%%%%%%%%%%%%%%%%%%%%%%%%%%%%%%%%%%%%%%%%%%%%%%%%%%%%
% user_check_validity_of_cut
%%%%%%%%%%%%%%%%%%%%%%%%%%%%%%%%%%%%%%%%%%%%%%%%%%%%%%%%%%%%%%%%%%%%%%%%%%%%%
\label{user_check_validity_of_cut}
\functiondef{user\_check\_validity\_of\_cut}
\begin{verbatim}
int user_check_validity_of_cut(void *user, cut_data *new_cut)
\end{verbatim}

\bd

\describe

This function is provided as a debugging tool. Every cut that is to be
sent to the LP solver is first passed to this function where the user
can independently verify that the cut is valid by testing it against a
known feasible solution (usually an optimal one). This is useful for
determining why a particular known feasible (optimal) solution was
never found. Usually, this is due to an invalid cut being added. See
Section \ref{debugging} for more on this feature.

\args

\bt{llp{280pt}}
{\tt void *user} & IN & Pointer to the user-defined data structure. \\
{\tt cut\_data *new\_cut} & IN & Pointer to the cut that must be
checked. \\
\et

\returns

\bt{lp{300pt}}
{\tt ERROR} & Ignored. \\
{\tt USER\_NO\_PP} & The user is done checking the cut. \\
\et

\item[Invoked from:] Whenever a cut is being sent to the LP.

\ed

\vspace{1ex}

%begin{latexonly}
\ed
%end{latexonly}

\newpage

\input{man-cp}

%\newpage

%%%%%%%%%%%%%%%%%%%%%%%%%%%%%%%%%%%%%%%%%%%%%%%%%%%%%%%%%%%%%%%%%%%%%%%%%%%%%
\subsection{User-written functions of the Draw Graph process}

Due to the relative simplicity of the cut pool, there are no wrapper
functions implemented for DG. Consequently, there are no default
options and no post-processing.

%begin{latexonly}
\bd
%end{latexonly}

%%%%%%%%%%%%%%%%%%%%%%%%%%%%%%%%%%%%%%%%%%%%%%%%%%%%%%%%%%%%%%%%%%%%%%%%%%%%%
% user_dg_process_message
%%%%%%%%%%%%%%%%%%%%%%%%%%%%%%%%%%%%%%%%%%%%%%%%%%%%%%%%%%%%%%%%%%%%%%%%%%%%%

\functiondef{user\_dg\_process\_message}
\begin{verbatim}
void user_dg_process_message(void *user, window *win, FILE *write_to)
			     
\end{verbatim}

\bd

\describe

The user has to process whatever user-defined messages are sent to the
process. A write-to pipe to the wish process is provided so that the
user can directly issue commands there.

\args

\bt{llp{250pt}}
{\tt void *user} & INOUT & Pointer to the user-defined data structure. \\
{\tt window *win} & INOUT & The window that received the message. \\
{\tt FILE *write\_to} & IN & Pipe to the wish process. \\
\et

\returns

\bt{lp{300pt}}
{\tt ERROR} & Error. Message ignored. \\
{\tt USER\_NO\_PP} & The user processed the message. \\
\et

\ed

\vspace{1ex}


%%%%%%%%%%%%%%%%%%%%%%%%%%%%%%%%%%%%%%%%%%%%%%%%%%%%%%%%%%%%%%%%%%%%%%%%%%%%%
% user_dg_init_window
%%%%%%%%%%%%%%%%%%%%%%%%%%%%%%%%%%%%%%%%%%%%%%%%%%%%%%%%%%%%%%%%%%%%%%%%%%%%%

\functiondef{user\_dg\_init\_window}
\begin{verbatim}
void user_dg_init_window(void **user, window *win) 
\end{verbatim}

\bd

\describe

The user must perform whatever initialization is necessary for
processing later commands. This usually includes setting up the user's
data structure for receiving and storing display data.

\args

\bt{llp{275pt}}
{\tt void **user} & INOUT & Pointer to the user-defined data structure. \\
{\tt window *win} & INOUT & \\
\et

\returns

\bt{lp{300pt}}
{\tt ERROR} & Error. Ignored. \\
{\tt USER\_NO\_PP} & The user successfully performed initialization. \\
\et

\ed

\vspace{1ex}


%%%%%%%%%%%%%%%%%%%%%%%%%%%%%%%%%%%%%%%%%%%%%%%%%%%%%%%%%%%%%%%%%%%%%%%%%%%%%
% user_dg_free_window
%%%%%%%%%%%%%%%%%%%%%%%%%%%%%%%%%%%%%%%%%%%%%%%%%%%%%%%%%%%%%%%%%%%%%%%%%%%%%

\functiondef{user\_dg\_free\_window}
\begin{verbatim}
void user_dg_free_window(void **user, window *win) 
\end{verbatim}

\bd

\describe

The user must free any data structures allocated.

\args

\bt{llp{275pt}}
{\tt void **user} & INOUT & Pointer to the user-defined data structure. \\
{\tt window *win} & INOUT & \\
\et

\returns

\bt{lp{300pt}}
{\tt ERROR} & Error. Ignored. \\
{\tt USER\_NO\_PP} & The user successfully freed the data structures. \\
\et

\ed

\vspace{1ex}

%%%%%%%%%%%%%%%%%%%%%%%%%%%%%%%%%%%%%%%%%%%%%%%%%%%%%%%%%%%%%%%%%%%%%%%%%%%%%
% user_interpret_text
%%%%%%%%%%%%%%%%%%%%%%%%%%%%%%%%%%%%%%%%%%%%%%%%%%%%%%%%%%%%%%%%%%%%%%%%%%%%%

\functiondef{user\_interpret\_text}
\begin{verbatim}
void user_interpret_text(void *user, int text_length,
			 char *text, int owner_tid)
\end{verbatim}

\bd

\describe

The user can interpret text input from the window.

\args

\bt{llp{250pt}}
{\tt void *user} & INOUT & Pointer to the user-defined data structure. \\
{\tt int text\_length} & IN & The length of {\tt text}. \\
{\tt char *text} & IN &  \\
{\tt int owner\_tid} & IN & The tid of the process that initiated this window.
\\ 
\et

\returns

\bt{lp{300pt}}
{\tt ERROR} & Error. Ignored. \\
{\tt USER\_NO\_PP} & The user successfully interpreted the text. \\
\et

\ed

\vspace{1ex}

%begin{latexonly}
\ed
%end{latexonly}



%\newpage

%\section{Data Structures}

%\input{man-data-BB}

%\input{man-data-lp}

%\input{man-data-cg}

%\input{man-data-cp}

%\input{man-data-tm}

%\input{man-data-dg}

%\newpage

\newpage

\section{\BB\ Parameters}
\label{params}
\input{man-param}

\newpage

\section{Bibliography}

\input{biblio}

%\newpage
%\section{Constant Names}

%\input{man-consts}

\end{document}
